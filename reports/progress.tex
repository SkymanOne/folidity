%% ----------------------------------------------------------------
%% Progress.tex
%% ---------------------------------------------------------------- 

\documentclass[oneside]{ecsproject}     % Use the Project Style
\usepackage{titlesec}
\usepackage{makecell}
\usepackage{pgfgantt}
\usepackage{pdflscape}
\titleformat{\chapter}
  {\normalfont\huge\bfseries}{\ \thechapter.}{18pt}{\huge}
\graphicspath{{figures/}}   % Location of your graphics files
\usepackage{natbib}            % Use Natbib style for the refs.
\hypersetup{colorlinks=false}   % Set to false for black/white printing

\newcommand{\vref}[1]{\textit{\hyperref[#1]{#1}}}

\begin{document}
\frontmatter
\title      {Folidity - Safe Functional Smart Contract Language}
\authors    {\texorpdfstring
             {\href{mailto:gn2g21@soton.ac.uk}{German Nikolishin}}
             {German Nikolishin}
            }
\addresses  {\groupname\\\deptname\\\univname}
\date       {\today}
\subject    {}
\keywords   {}
\supervisor {\texorpdfstring
            {\href{mailto:gn2g21@soton.ac.uk}{Prof. Vladimiro Sassone}}
            {Prof. Vladimiro Sassone}
}
\examiner   {Dr Butthead}
\degree     {BSc Computer Science}
\maketitle
\tableofcontents
% \listoffigures
% \listoftables
% \lstlistoflistings
% \listofsymbols{ll}{$w$ & The weight vector}
\mainmatter
%% ----------------------------------------------------------------

\chapter{Introduction}

The idea of "smart contract" [SC] was first coined by Nick Szabo as a computerised transaction protocol \cite{nz_sc}.
He later defined smart contracts as observable, verifiable, privity-applicable, and enforceable programs. \cite{nz_sc_bb}.
In other words, it was envisioned for smart contracts to inherit the natural properties of the traditional
"paper-based" contracts.

It was only in 2014 when SCs were technically formalised at the protocol level, as an arbitrary program
written in some programming language (Solidity) and executed in the blockchain's virtual machine (EVM) \cite{eth_yellow_paper}.

Ethereum Virtual Machine (EVM) iterated over the idea of Bitcoin Scripting allowing developers to deploy general-purpose, Turing-Complete
programs that can have own storage, hence state. This enabled the development of more sophisticated applications that spanned beyond
the simple transfers of funds among users.

Overall, SC can be summarised as an \textit{immutable}, \textit{permissionless}, \textit{deterministic} computer programs 
that are executed as part of state transition in the blockchain system. 
At the time of writing, Solidity is still the most widely used SC language (SCL) \cite{sc_survey}.

After a relatively short time, SCs have come a long way and allowed users to access different online services in a completely trustless and decentralised way.
The applications have spanned across financial, health, construction and many other sectors. 

\chapter{Security and Safety of Smart Contracts}

\section{Overview}

With the increased adoption of decentralised applications (DApps) and the increased total value locked in DApps, 
there has been evidence of numerous attacks and exploits focused on extracting funds from SCc in an unconventional way. 
Due to the permissionless nature of SCs, the most common way of attacks is by exploiting the mistakes in the SC's source code.
Specifically, the attacker can not tamper with the protocol code due to consensus mechanisms. 
Instead, they can cleverly tamper with the publicly accessible parameters to force the SC into an unexpected state, essentially gaining partial control of it.

A notorious example of such attacks is the DAO hack when hackers exploited unprotected re-entrance calls to withdraw \textbf{\$50 million worth of ETH}. 
This event forced the community to hard-fork the protocol to revert the transaction provoking a debate on the soundness of the action \cite{the_dao}.

Another less-known example is the "King of the Ether" attack which was caused by the unchecked low-level Solidity \texttt{send} call to transfer funds to some account \cite{king_of_the_ether}.

Other issues involve the \textit{safety} and \textit{usability} of SCs. Due to programmer mistakes, SCs can enter an unexpected state preventing its intended functionality \cite{ondo_report}.


\section{Vulnerability classification}

There has been an effort in both academia and industry to classify common vulnerabilities 
and exploits in SCs in blockchain systems \cite{owasp}\cite{stefano}\cite{atzei_survey}. 
Some of the work has been recycled by bug bounty platforms growing the community of auditors
and encouraging peer-review of SCs such as "code4rena"\footnote{https://code4rena.com}, "Solodit"\footnote{https://solodit.xyz},
and many others.

Analysing the work mentioned above, SCs vulnerabilities can be categorised into the 6 general groups that are outlined in \tref{Table:classification}

\begin{table}[!htb]
  \centering
  \begin{tabular}{ccc}
  \toprule
  \textbf{Code} & \textbf{Title} & \textbf{Summary}\\
  \midrule
  \textit{SCV1}\label{SCV:1} & \makecell{Timestamp\\manipulation} & \makecell{\\Timestamp used in \\control-flow, randomness and storage,\\can open an exploit due to an ability\\for validator to manipulate the timestamp}\\\\
  \hline
  \textit{SCV2}\label{SCV:2} & \makecell{Pseudo-randomness} & \makecell{\\Using block number, block hash,\\block timestamp\\are not truly random generated parameters,\\and can be manipulated by the adversary validator}\\\\
  \hline
  \textit{SCV3}\label{SCV:3} & \makecell{Invalidly-coded\\states} & \makecell{\\When coding business logic,\\ control-flow checks\\can be incorrectly coded resulting the SC\\entering into invalid state}\\\\
  \hline
  \textit{SCV4}\label{SCV:4} & \makecell{Access Control\\exploits} & \makecell{\\This is a more categorisation of vulnerabilities.\\It occurs when an adversary calls a restricted function.\\This is specifically present in\\\textit{upgradeability} and \textit{deleteability} of SCs}\\\\
  \hline
  \textit{SCV5}\label{SCV:5} & \makecell{Arithmetic operations} & \makecell{\\SCs are suspected to the same arithmetic bugs\\as classic programs.\\Therefore, unchecked operations can result\\in underflow/overflow or deletion by zero}\\\\
  \hline
  \textit{SCV6}\label{SCV:6} & \makecell{Unchecked externall\\calls} & \makecell{\\Unchecked re-entrant, forward, delegate\\calls can result in the contract\\entering into unexpected state}\\\\
  \bottomrule
  \end{tabular}
  \caption{classification of SC Vulnerabilities}
  \label{Table:classification}
\end{table}

Note that we do not evaluate the listed vulnerabilities based on their severity. 
As far as this paper is concerned, all vulnerabilities are considered to be of equal weight for the reasons described in \sref{Section:Scene}.

\newpage
\section{Setting the scene} \label{Section:Scene}

Numerous deployed DApps allowed the community of developers and auditors to learn from the mistakes and the past,
and generally improve the code quality and security of SCs. Audits are now an essential part of the release cycle of any DApp.

However, even with the raised awareness for the security and safety of SCs, recent reports from "code4rena" still show \vref{SCV:3}, \vref{SCV:4} and \vref{SCV:5}
present in the recent audit reports\cite{arcade_report}\cite{ondo_report}\cite{centrifuge_report}.

In particular, in \cite{centrifuge_report} a relatively simple calculation mistake resulted in other SC users being unable to withdraw their funds.

It can be seen that SC Vulnerabilities illustrated in \tref{Table:classification} are still evident in modern SCs resulting in opening them up to vulnerabilities of different severity levels.
Therefore, as mentioned earlier, we can not classify any particular vulnerability to be more severe than the other as it solely depends on the context in the code it is present in.

Furthermore, given the pattern in the mistakes made by SC developers, 
it has been realised that additional tooling or alternative SCLs need to be discovered to minimise the exposure of SC code to the earlier-mentioned vulnerabilities.

\chapter{Current Solutions}

\section{Overview}

As mentioned earlier, given the increased use of SCs and the consistency in the presence of vulnerabilities and programmer mistakes
different solutions have been presented to mitigate those. 
We can generally categorise them into 2 groups: safe SCLs which allow users to write safe and secure code, particularly described in \cref{Chapter:SCL}, 
and formal verification tools which are used alongside traditional SCLs and are described in \cref{Chapter:FVT}.

At the end of the chapter, we will have reviewed both categories of tools allowing us to evaluate their effectiveness in correlation to usability.
Particularly, this chapter aims to provide a clear and concise framework to analyze and work with the SC tools dedicated to producing
error-proof DApps. 
\section{Safe Smart Contract Languages} \label{Chapter:SCL}

\section{Formal Verification Tools} \label{Chapter:FVT}

\section{Summary}

\chapter{Proposed Solution} \label{Chapter:Solution}

%% ----------------------------------------------------------------

\chapter{Project Planning}




%% ----------------------------------------------------------------

\appendix

\chapter{Gannt Chart}
\begin{landscape}
  \begin{figure}[tbp]
      \centering
      \begin{ganttchart}[y unit title=0.5cm,
      y unit chart=0.5cm,
      vgrid,hgrid,
      % title label anchor/.style={below=-1.6ex},
      % title left shift=.05,
      % title right shift=-.05,
      title height=1.5,
      % progress label text={Progress},
      bar height=0.8,
      % group right shift=0,
      group top shift=.8,
      ]{1}{30}
      %labels
      \gantttitle{\bf{Month}}{30} \\ \\
      \gantttitle{October}{4} 
      \gantttitle{November}{4} 
      \gantttitle{December}{4} 
      \gantttitle{January}{4} 
      \gantttitle{February}{4} 
      \gantttitle{March}{4}
      \gantttitle{April}{4}
      \gantttitle{May}{2} \\
      %tasks
      \ganttgroup{Overview of SC vulnerabilities}{1}{9} \\
      \ganttbar{Common smart contract exploits}{2}{7} \\
      \ganttbar{Survey of SC languages}{3}{8} \\
      \ganttbar{Formal verification analysis}{6}{9} \\
      \ganttbar{Evaluation of current solutions and issues}{6}{8} \\ \\
  
      \ganttmilestone{Progress Report}{9} \\
  
      \ganttgroup{The design of the proposal solution}{7}{14} \\
      \ganttbar{Requirements}{7}{8} \\
      \ganttbar{BNF Grammar}{7}{9} \\
      \ganttbar{Sample programs}{9}{12} \\
      \ganttbar{Brief evaluation of the BNF grammar}{12}{14} \\ \\
  
      \ganttgroup{Implementation}{12}{24} \\
      \ganttbar{Syntax}{12}{14} \\
      \ganttbar{Type Checker}{15}{18} \\
      \ganttbar{Evaluator}{18}{24} \\
      \ganttbar{Model Checker}{18}{21} \\
      \ganttbar{Functional correctness checker}{20}{23} \\ \\
  
      \ganttgroup{Evaluation}{23}{28} \\ 
      \ganttbar{Testing}{23}{26} \\
      \ganttbar{Safety evaluation}{27}{27} \\
      \ganttbar{Overview of sample solutions}{27}{27} \\
      \ganttbar{Future work}{28}{28} \\ \\
  
      \ganttmilestone{Submission Deadline}{28}
      %relations 
  
      \end{ganttchart}
      \caption{Gantt Chart}
  
  \end{figure}
  \end{landscape}

%% ----------------------------------------------------------------


\backmatter
\bibliographystyle{unsrt}
\bibliography{ECS}
\include{AppendixA}
\end{document}
%% ----------------------------------------------------------------
